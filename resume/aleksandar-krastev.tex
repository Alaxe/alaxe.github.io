\documentclass[letterpaper,11pt]{article}

\usepackage[letterpaper, left=1in, right=1in, bottom=1in, top=1in]{geometry}

\usepackage{enumitem}
\usepackage{titlesec}
\usepackage{hyperref}

\usepackage{noto}
\usepackage[T1]{fontenc}
\setmainfont[BoldFont={Noto Serif Medium}]{Noto Serif Light}


\setlength{\parindent}{0em}
\setlength{\parskip}{.5em}
\titlespacing{\section}{0pt}{0em}{0em}
\titleformat{\section}{\bf\Large}{\thesection}{1em}{}[{\titlerule[0.8pt]}]
\setlist{nosep, topsep=-0.4em}
\setlist[description]{leftmargin=!}

\renewcommand{\labelitemi}{\textbullet}
\renewcommand{\labelitemii}{$\circ$}
\renewcommand{\thesection}{}

\pagenumbering{gobble}

\author{Aleksandar Krastev}
\date{2020-12-21}

\begin{document}
\setlength{\belowdisplayskip}{\parskip}
\setlength{\belowdisplayshortskip}{\belowdisplayskip}
\setlength{\abovedisplayskip}{\parskip}
\setlength{\abovedisplayshortskip}{\abovedisplayskip}

\centerline{{\Huge \bf Aleksandar Krastev}}
$\bullet$ \href{mailto:alexalex@mit.edu}{alexalex@mit.edu} \hfill
$\bullet$ \href{https://alexalex.xyz}{https://alexalex.xyz} \hfill
$\bullet$ \href{https://github.com/Alaxe}{https://github.com/Alaxe} \hfill

\section*{Education}
\textbf{Massachusetts Institute of Technology}
\hfill
\textsc{Expected Jun 2023}

\begin{itemize}
    \item Pursuing M.Eng. and B.S. in Computer Science
        \hfill \textsc{GPA 5.0/5.0}
    \item Past courses:
        Computer System Architecture,
        Secure Hardware Design,
        Advanced Data Structures,
        Distributed Systems,
        Theory of Computation,
        Software Construction,
        Algorithms for Graphs and Matrices.
\end{itemize}



\section*{Experience}
\textbf{MIT / Graduate Research Assistant}
\hfill
\textsc{Jun 2022--Present}
\begin{itemize}
    \item Co-lead the development of an efficient, easy-to use FHE compiler
        (\emph{under submission}).
\end{itemize}

\textbf{MIT / Undergraduate Researcher}
\hfill
\textsc{Jan 2021--May 2022}
\begin{itemize}
    \item Helped design the state-of-the-art FHE hardware accelerators
        (\emph{CraterLake} and \emph{F1}).
\end{itemize}

\textbf{Amazon.com / Software Engineering Intern}
\hfill
\textsc{Jun 2021--Aug 2021}
\begin{itemize}
    \item Worked on the Amazon Halo mobile app using React Native and TypeScript.
\end{itemize}

\textbf{QuantCo / Software Engineering Intern}
\hfill
\textsc{Jun 2020--Aug 2020}
\begin{itemize}
    \item Migrated \href{https://grf-labs.github.io/}{Random Forests} model
        training and validation from R to Python using
        \href{https://rpy2.github.io/}{rpy2}.
\end{itemize}

\textbf{Bulgarian Informatics / Instructor \& Problem Author}
\hfill
\textsc{Aug 2017--Jul 2020}
\begin{itemize}
    \item Developed 2 tasks for contests with 75+ participants from 7 countries.
    \item Prepared 3 talks on data structures and algorithms for Bulgaria's
        top 20 students.
\end{itemize}

\section*{Publications}
\begin{description}[style=sameline]
\item[\href{https://dl.acm.org/doi/pdf/10.1145/3470496.3527393}
    {CraterLake: A Hardware Accelerator for Efficient Unbounded Computation on
    Encrypted Data.}]
    N.Samardzic, A. Feldmann, \textbf{A. Krastev}, N. Manohar, N. Genise, S.
    Devadas, K. Eldefrawy, C. Peikert, D. Sanchez, \textit{49th
    International Symposium in Computer Architecture (ISCA-48), June 2022}
\item[\href{https://dl.acm.org/doi/pdf/10.1145/3466752.3480070}
    {F1: A Fast and Programmable Accelerator for Fully Homomorphic Encryption.}]
    A. Feldmann, N. Samardzic, \textbf{A. Krastev}, S. Devadas, R. Dreslinski,
    C. Peikert, D. Sanchez, \textit{54th annual IEEE/ACM international
    symposium on Microarchitecture (MICRO-54), October 2021}\\
    \textbf{(IEEE Micro's Top Pick for 2022)}
\end{description}

\section*{Independent Projects}
\begin{description}[labelwidth=3.75em]
    \item[\href{https://github.com/Alaxe/nitwit}{Nitwit}]
        Compiler for a made-up programming language (C++)
    \item[\href{https://github.com/6851-2021/retroactive-priority-queue}{retropq}]
        Priority queue that can change past operations in $O(\log n)$
        (Python)
    \item[\href{https://github.com/Alaxe/gemini}{Gemini}]
        Co-op puzzle platformer with real-time multiplayer (JavaScript)
    \item[\href{https://github.com/Alaxe/stealth}{Stealth}]
        2D game of sneaking past guards; computes visible regions in $O(n \log n)$
        (JavaScript)
\end{description}

\section*{Awards}
\begin{description}[labelwidth=6.25em]
    \item[Gold medal] (24th of 335) International Olympiad in Informatics, 2018
    \item[Silver medal] (29th of 160) European Physics Olympiad, 2019
\end{description}
\end{document}
